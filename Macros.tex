\usepackage{a4wide,times,amsmath,amssymb,eurosym,graphicx,mathtools}
\usepackage{amsthm}
\usepackage{bbm}	%Für \mathbbm
\usepackage{enumitem}
\usepackage{hyperref}
\usepackage[german,english]{babel}
\usepackage{comment}
\usepackage{betterStackrel}
\usepackage{csquotes}
\parindent0pt
\parskip0.5ex

\newtheorem*{remark}{Remark}
\theoremstyle{definition}
\newtheorem{exercise}{Exercise}
\theoremstyle{plain}
\newtheorem*{theorem}{Theorem}
\newtheorem*{definition}{Definition}

%%%% define \begin{varthm}{<reference to \label{} of other theorem>}[<Exercise|Theorem|Def|...>]
\newtheorem*{varthm+inner}{\varthmname}
\newcommand{\varthmname}{}
\newenvironment{varthm}[2][Exercise]
 {\renewcommand{\varthmname}{#1 \ref{#2}'}\begin{varthm+inner}}
 {\end{varthm+inner}}

\setlist[enumerate,1]{label={(\roman*)}}
\setlist[enumerate,2]{label={(\alph*)}}
\setlist[enumerate,3]{label={(\Roman*)}}

\newcommand{\bE}{\mathbb{E}}
\newcommand{\bQ}{\mathbb{Q}}
\newcommand{\bM}{\mathbb{M}}
\newcommand{\bN}{\mathbb{N}}
\newcommand{\bP}{\mathbb{P}}
\newcommand{\bR}{\mathbb{R}}
\newcommand{\bT}{\mathbb{T}}
\newcommand{\bU}{\mathbb{U}}
\newcommand{\bC}{\mathbb{C}}

\newcommand{\cE}{\mathcal{E}}
\newcommand{\cF}{\mathcal{F}}
\newcommand{\cG}{\mathcal{G}}
\newcommand{\cQ}{\mathcal{Q}}
\newcommand{\cP}{\mathcal{P}}
\newcommand{\cA}{\mathcal{A}}
\newcommand{\cD}{\mathcal{D}}
\newcommand{\cB}{\mathcal{B}}
\newcommand{\cU}{\mathcal{U}}
\newcommand{\cN}{\mathcal{N}}
\newcommand{\cL}{\mathcal{L}}
\newcommand{\Real}{\mathrm{Re}}
\newcommand{\Imag}{\mathrm{Im}}
\newcommand{\ind}[1]{\mathbbm{1}_{#1}} %Definition 

\newcommand{\Exp}[2][]{\bE_{ #1 }\left[ #2 \right]}
\renewcommand{\Pr}{\mathbb{P}}
\newcommand{\Cov}{\text{Cov}}
\newcommand{\Var}{\text{Var}}
\newcommand{\Poi}{\text{Poi}}
\newcommand{\Unif}{\cU}
\newcommand{\Normal}{\cN}
\newcommand{\charFct}{\varphi}
\newcommand{\integer}{\mathbb{Z}}
\newcommand{\naturals}{\mathbb{N}}
\newcommand{\conj}[1]{\overline{#1}}
\newcommand{\reals}{\mathbb{R}}
\newcommand{\complex}{\mathbb{C}}
%
\newcommand{\eps}{\varepsilon}
\renewcommand{\d}{\mathrm{d}}
\newcommand{\eqdist}{\overset{\mathrm{(d)}}{=}}
\newcommand{\as}{almost surely}
\newcommand{\dint}{\, \mathrm{d}}
\newcommand*{\indep}{{\perp \!\!\! \perp}}
\newcommand\restr[2]{\ensuremath{\left.#1\right|_{#2}}}

\renewcommand{\mid}{\,|\,}
\renewcommand{\complement}{c}

%%% better stackrel shorthands

\newcommand*{\lximplies}[1]{\lstackrel{#1}{\implies}} %%%%% =>
\newcommand*{\lximpliedby}[1]{\lstackrel{#1}{\impliedby}} %%%%%% <=
\newcommand*{\lxiff}[1]{\lstackrel{#1}{\iff}} %%%%%%%%%%%%% <=>
\newcommand*{\lxeq}[1]{\lstackrel{#1}{=}} %%%%%%%%%% =
\newcommand*{\lxle}[1]{\lstackrel{#1}{\le}}
\newcommand*{\lxge}[1]{\lstackrel{#1}{\ge}}

\newcommand*{\ximplies}[1]{\stackrel{#1}{\implies}} %%%%% =>
\newcommand*{\ximpliedby}[1]{\stackrel{#1}{\impliedby}} %%%%%% <=
\newcommand*{\xiff}[1]{\stackrel{#1}{\iff}} %%%%%%%%%%%%% <=>
\newcommand*{\xeq}[1]{\stackrel{#1}{=}} %%%%%%% =
\newcommand*{\xle}[1]{\stackrel{#1}{\le}} %%%%%%%
\newcommand*{\xge}[1]{\stackrel{#1}{\ge}} %%%%%%%
\newcommand*{\xsim}[1]{\stackrel{#1}{\sim}} %%%%%%%
\newcommand*{\xto}[1]{\stackrel{#1}{\to}} %%%%%%%

%%%%%%%%%%%%%%%%%%%%%%%%%%%%%%%%%%%%%%%%%%%%%%%%%%%%%%%
% Solution Toggle
%%%%%%%%%%%%%%%%%%

\ifsolution
	\newenvironment{solution}[1][{Solution}] {\begin{proof}[#1]} {\end{proof}}
\else
	\excludecomment{solution}
	\hinttrue % automatically activate hints when solutionfalse
\fi

\ifhint
	\newtheorem*{hint}{Hint}
\else
	\excludecomment{hint}
\fi

%%%%
